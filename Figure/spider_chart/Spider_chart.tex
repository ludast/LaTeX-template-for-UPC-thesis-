% Author: Ahmed Sabir (May 2024)
\documentclass[varwidth]{standalone}[2011/222/21]
\usepackage{pgfplots}
\renewcommand*{\familydefault}{\sfdefault}
\usepackage{verbatim}
\usepackage[active,tightpage,floats]{preview}
\setlength\PreviewBorder{5pt}%
\definecolor{BLUE-light}{rgb}{0.29, 0.59, 0.82}
\definecolor{BLUE-cu}{rgb}{0.0, 0.5, 0.40}
\definecolor{red-cu}{rgb}{0.9, 0.4, 0.38}
\definecolor{ao(english)}{rgb}{0.0, 0.5, 0.0}
\newcommand{\customfont}[2]{\fontsize{#1}{#2}\selectfont}
\usepackage{tikz}
\usetikzlibrary{shapes, calc}

\begin{document}

\newcommand{\D}{8} % Number of dimensions
\newcommand{\U}{6} % Number of levels

\newdimen\R % Maximal diagram radius
\R=3.5cm
\newdimen\L % Radius to put dimension labels
\L=4cm

\newcommand{\A}{360/\D} % Angle between dimension axes

\begin{figure}[htbp]
\centering

\begin{tikzpicture}[scale=1]
  \path (0:0cm) coordinate (O); % Center

  % Draw the spiderweb
  \foreach \X in {1,...,\D}{
    \draw (\X*\A:0) -- (\X*\A:\R);
  }

  % Draw concentric circles and radial lines
  \foreach \Y in {0,...,\U}{
    \foreach \X in {1,...,\D}{
      \path (\X*\A:\Y*\R/\U) coordinate (D\X-\Y);
      \fill (D\X-\Y) circle (1pt);
    }
    \draw [opacity=0.3, dotted, line width=1pt] (0:\Y*\R/\U) \foreach \X in {1,...,\D}{
        -- (\X*\A:\Y*\R/\U)
    } -- cycle;
  }
  \path (1*\A:\L) node (L1) {};
  % ... Add other labels ...


% Drawing additional concentric circles for decimal values
\foreach \Y in {0,...,\U}{
  \foreach \X in {1,...,\D}{
    \path (\X*\A:\Y*\R/\U) coordinate (D\X-\Y);
    \fill (D\X-\Y) circle (1pt);
  }
  % Draw main level circle
  \draw [opacity=0.3, dotted, line width=1pt] (0:\Y*\R/\U) \foreach \X in {1,...,\D}{
      -- (\X*\A:\Y*\R/\U)
  } -- cycle;

  % Add intermediary circle for decimal at 0.5
  \ifnum\Y<\U % Check to not draw beyond the last level
    \pgfmathsetmacro{\DecimalY}{\Y + 0.5}
    \draw [opacity=0.2, dotted, line width=0.7pt, color=gray] (0:\DecimalY*\R/\U) \foreach \X in {1,...,\D}{
        -- (\X*\A:\DecimalY*\R/\U)
    } -- cycle;
  \fi
}

  % Adding numbers 
  \foreach \Y in {1,...,\U}{
    \node[fill=white, inner sep=1pt, font=\tiny] at (0:\Y*\R/\U) {\Y}; % Change 0 to \X*\A for numbers on all radial lines
  }

  

  \path (1*\A:\L) node (L1) {\small A};

  \path (2*\A:\L*0.95) node (L2) {\small B}; 
  \path (3*\A:\L) node (L3) {\small C};
  %\path (4*\A:\L) node (L4) {\small Netflix\textcolor{white}{.}};
   \path (4*\A:\L*1.02) node (L4) {\small D}; %
  \path (5*\A:\L) node (L5) {\small E};
%  \path (6*\A:\L) node (L6) {\small Spotify};
  \path (6*\A:\L*0.95) node (L6) {\small F}; %
  \path (7*\A:\L) node (L7) {\small G};
  \path (8*\A:\L) node (L8) {\small E};
  

  

\filldraw[fill=BLUE-light, fill opacity=0.3, draw opacity=0.5, line width=1.5pt]
    % ((1∗\A:0)!5.8/\U!(1∗\A:\R)(1*\A:0)!5.8/\U!(1*\A:\R)) -- % The problem is that the position needs to be calculated manually for 0.5 (5.5) on the first axis .. because the plotting package does not support it 
    ($(1*\A:0)!3.75/\U!(1*\A:\R)$) --
    ($(2*\A:0)!3.47/\U!(2*\A:\R)$) --
     ($(3*\A:0)!2.89/\U!(3*\A:\R)$) --
     ($(4*\A:0)!2.87/\U!(4*\A:\R)$) --
     ($(5*\A:0)!3.28/\U!(5*\A:\R)$) --
     ($(6*\A:0)!4.92/\U!(6*\A:\R)$) --
     ($(7*\A:0)!5.29/\U!(7*\A:\R)$) --
     ($(8*\A:0)!4.59/\U!(8*\A:\R)$) -- cycle;


\filldraw[fill=BLUE-cu, fill opacity=0.3, draw opacity=0.5, line width=1.5pt]
    ($(1*\A:0)!4.25/\U!(1*\A:\R)$) --
    ($(2*\A:0)!3.69/\U!(2*\A:\R)$) --
     ($(3*\A:0)!3.50/\U!(3*\A:\R)$) --
     ($(4*\A:0)!3.13/\U!(4*\A:\R)$) --
     ($(5*\A:0)!3.90/\U!(5*\A:\R)$) --
     ($(6*\A:0)!3.93/\U!(6*\A:\R)$) --
     ($(7*\A:0)!4.19/\U!(7*\A:\R)$) --
     ($(8*\A:0)!3.65/\U!(8*\A:\R)$) -- cycle;

\filldraw[fill=red-cu, fill opacity=0.3, draw opacity=0.5, line width=1.5pt]
    ($(1*\A:0)!3.95/\U!(1*\A:\R)$) --
    ($(2*\A:0)!3.61/\U!(2*\A:\R)$) --
     ($(3*\A:0)!2.81/\U!(3*\A:\R)$) --
     ($(4*\A:0)!2.92/\U!(4*\A:\R)$) --
     ($(5*\A:0)!3.82/\U!(5*\A:\R)$) --
     ($(6*\A:0)!3.61/\U!(6*\A:\R)$) --
     ($(7*\A:0)!3.75/\U!(7*\A:\R)$) --
     ($(8*\A:0)!3.11/\U!(8*\A:\R)$) -- cycle;



\matrix [below=10pt, every node/.style={anchor=west, font=\small}, column sep=5mm, row sep=0mm, anchor=north, inner sep=5pt, fill=white, rounded corners] at (current bounding box.south) {
    \draw [color=BLUE-light, line width=3pt, opacity=0.7] (0,0) -- ++(0.5,0) node[right=0.1mm]{Model A}; &
    \draw [color=BLUE-cu, line width=3pt, opacity=0.7] (0,0) -- ++(0.5,0) node[right=0.1mm]{Model B}; &
    \draw [color=red-cu, line width=3pt, opacity=0.7] (0,0) -- ++(0.5,0) node[right=0.1mm]{Model C}; \\
    };

\end{tikzpicture}
%%\caption{}
\label{fig:spiderweb}
\end{figure}

\end{document}
